%%% rt1d

\documentclass[letterpaper,titlepage,12pt]{article}

%%% Preamble 

\setlength{\topmargin}{0in}
\setlength{\oddsidemargin}{0in}
\setlength{\evensidemargin}{0in}
\setlength{\textwidth}{6.5in}
\setlength{\textheight}{9in}
\setlength{\headheight}{0in}
\setlength{\headsep}{0in}
\setlength{\marginparsep}{0in}
\setlength{\marginparwidth}{0in}

\usepackage{amsmath, amsthm, amssymb}
\numberwithin{equation}{section}
\usepackage{graphicx}
\usepackage{pslatex}
\usepackage{natbib}
\usepackage[pdfborder={0 0 0}]{hyperref}
\usepackage[all]{hypcap}

%%% Beginning of Document

\begin{document}
    
%%% START custom commands
\newcommand{\HI}{\text{HI}}
\newcommand{\HII}{\text{HII}}
\newcommand{\HeI}{\text{HeI}}
\newcommand{\HeII}{\text{HeII}}
\newcommand{\HeIII}{\text{HeIII}}
\newcommand{\nH}{n_{\text{H}}} 
\newcommand{\nHI}{n_{\text{HI}}} 
\newcommand{\nHII}{n_{\text{HII}}}  
\newcommand{\nHeI}{n_{\text{HeI}}}
\newcommand{\nHeII}{n_{\text{HeII}}}
\newcommand{\nHeIII}{n_{\text{HeIII}}}  
\newcommand{\nel}{n_{\text{e}}}  
\newcommand{\nb}{n_{\text{b}}}
\newcommand{\nnu}{$n_{\nu}$ }
\newcommand{\ncol}{n_i^{\text{col}}}
\newcommand{\gHI}{\Gamma_{\text{HI}}}  
\newcommand{\gHeI}{\Gamma_{\text{HeI}}}
\newcommand{\gHeII}{\Gamma_{\text{HeII}}}
\newcommand{\aHII}{\alpha_{\text{HII}}}  
\newcommand{\aHeII}{\alpha_{\text{HeII}}}  
\newcommand{\aHeIII}{\alpha_{\text{HeIII}}}
\newcommand{\bHI}{\beta_{\text{HI}}} 
\newcommand{\bHeI}{\beta_{\text{HeI}}}  
\newcommand{\bHeII}{\beta_{\text{HeII}}}  
\newcommand{\xHeII}{\xi_{\text{HeII}}}
\newcommand{\kB}{k_{\text{B}}}
\newcommand{\fheat}{f_i^{\text{heat}}}
\newcommand{\fion}{f_i^{\text{ion}}}
\newcommand{\Lbol}{\mathcal{L}_{\text{bol}}}
\newcommand{\spec}{\mathcal{N}}
\newcommand{\Heat}{\mathcal{H}}
\newcommand{\trec}{$t_{\text{rec}}$}
\newcommand{\Lbox}{L_{\mathrm{box}}}
\newcommand{\dx}{\Delta x}
\newcommand{\dd}{\text{d}}
\newcommand{\Htwo}{\mathrm{H}_2}
\newcommand{\eV}{\mathrm{eV}}
%%% END custom commands
	
\author{Jordan Mirocha}	
	
\title{\Large {\bf rt1d: User's Manual}}
\date{Last Updated: \today}
\maketitle

\setcounter{tocdepth}{2}
\tableofcontents
\newpage

%%%
%% Introduction
%%%
\section{Introduction}
Let's see what happens to the gas around stars and black holes, shall we?

%%% 
%% Methods
%%%
\section{Methods}
Our radiative transfer scheme is modeled closely after the work of
\cite{Fukugita1994} and \cite{Thomas2008}, though for completeness we will
reiterate the aspects of these methods most pertinent to the problem at hand.
In general, propagating a radiation field relies on solving differential
equations governing the rate of change in the number densities of all ions and
the temperature of the gas. Assuming a medium consisting of hydrogen and
helium only, we first solve for the relative abundances of each ion via
\begin{align}
    \frac{d \nHII}{dt} & = \gHI \nHI + \bHI \nel \nHI - \aHII \nel \nHII   \label{eq:HIIRateEquation} \\ 
    \frac{d \nHeII}{dt} & = \gHeI \nHeI + \bHeI \nel \nHeI - \bHeII \nel \nHeII - \aHeII \nel \nHeII + \aHeIII \nel \nHeIII - \xHeII \nel \nHeIII  \label{eq:HeIIRateEquation} \\ 
    \frac{d \nHeIII}{dt} & = \gHeII \nHeII + \bHeII \nel \nHeII - \aHeIII \nel \nHeIII . \label{eq:HeIIIRateEquation}
\end{align}
Each of these equations represents the balance of ionizations and
recombinations for ions $\HII$, $\HeII$, $\HeIII$. We associate the index $i$
with neutral species, $i = \HI, \HeI, \HeII$, and define $\Gamma_i$ as the
photo-ionization rate coefficient, $\alpha_i$ ($\xi_i$) as the case-A
(dielectric) recombination rate coefficients, $\beta_i$ as the collisional
ionization rate coefficients, and $\nel = \nHII + \nHeII + 2\nHeIII$ as the number density of electrons.

In this work, we use the formulae in Appendix B of \cite{Fukugita1994} to
compute the values of $\alpha_i$, $\beta_i$, and $\xi_i$. As you might expect, collisional ionization and recombination depend solely on the temperature of the gas.  For example, collisional ionization of hydrogen (from $n = 1$) is governed by
\begin{equation}
    \beta_{1,\HI} = 5.85 \times 10^{-11} T^{1/2} \left[1 + \left(\frac{T}{10^5 \mathrm{K}}\right)^{1/2}\right]^{-1} \mathrm{exp}\left[-1.578 \times \left(\frac{T}{10^5 \mathrm{K}}\right)^{-1}\right] \mathrm{cm^3 s^{-1}}
\end{equation}
and recombinations (to $n \ge 1$) by
\begin{equation}
    \alpha_{\HII} = 6.28 \times 10^{-11} T^{-1/2} \left(\frac{T}{10^3\mathrm{K}}\right)^{-1/5} \left[1 + \left(\frac{T}{10^6\mathrm{K}}\right)^{0.7}\right]^{-1} \mathrm{cm^3s^{-1}}
\end{equation}

We absorb ionizations due to to energetic photo-electrons into the
photoionization rate coefficients, $\Gamma_i$, which are given by
\begin{equation}
    \Gamma_i = \int_{E_i}^{\infty} \sigma_i(E) \spec(E) \frac{dE}{E} + \sum_j \fion \left(\frac{n_j}{n_i}\right) \int_{E_i}^{\infty} \sigma_i(E)\left(\frac{E - E_j}{E_i}\right) \spec(E) \frac{dE}{E} \label{eq:PhotoionizationRate}
\end{equation}
where $E_i$, $\sigma_i$, $n_i$, and $f_i$ are the ionization threshold energy,
bound-free absorption cross section, number density, and fraction of
photo-electron energy that goes into ionization of species $i$, respectively.
The term summing over $j$, where $j = \HI, \HeI, \HeII$, represents
ionizations of species $i$ due to fast secondary electrons from
photoionizations of species $j$, which has number density $n_j$. $\spec$ is
the spectral intensity, which we write as
\begin{equation}
    \spec(E, r) = \frac{\Lbol}{4\pi r^2} I(E) e^{-\tau(E, r)} . \label{eq:Spectrum}
\end{equation}
Here, $r$ is the distance from the source, $I(E)$ is the normalized spectral
energy distribution (SED), satisfying
\begin{equation}
    \int_E I(E) dE = 1 ,
\end{equation}
and the optical depth $\tau(E, r)$ is given as a sum over absorbing species,
\begin{align}
    \tau(E, r) & = \sum_i \int_r \sigma_i(E) n_i(r) dr \nonumber \\
               & = \sum_i \sigma_i(E) \ncol(r) \label{eq:OpticalDepth}
\end{align}
where $\ncol(r)$ is the column density of species $i$ at distance $r$ from the
source. We calculate the bound-free absorption cross-sections, $\sigma_i$,
using the fits of \cite{Verner1996}. 

To be more explicit, let's write out $\Gamma_{\HI}$ in its entirety.
\begin{align}
    \Gamma_{\HI} & = \int_{13.6\eV}^{\infty}\sigma_{\HI}(E) \spec(E) \frac{dE}{E} \nonumber \\
    & + f_{\HI}^{\mathrm{ion}} \left(\frac{n_{\HI}}{n_{\HI}}\right) \int_{13.6\eV}^{\infty}\sigma_{\HI}(E) \left(\frac{E - 13.6\eV}{E}\right)\spec(E)\frac{dE}{E} \nonumber \\
    & + f_{\HeI}^{\mathrm{ion}} \left(\frac{n_{\HeI}}{n_{\HI}}\right) \int_{24.4\eV}^{\infty}\sigma_{\HeI}(E) \left(\frac{E - 24.4\eV}{E}\right)\spec(E)\frac{dE}{E} \nonumber \\
    & + f_{\HeII}^{\mathrm{ion}} \left(\frac{n_{\HeII}}{n_{\HI}}\right) \int_{54.4\eV}^{\infty}\sigma_{\HeI}(E) \left(\frac{E - 54.4\eV}{E}\right)\spec(E)\frac{dE}{E}
\end{align}    
The first term accounts for all hydrogen photoionization.  The final three terms account for collisional ionization of hydrogen due to secondary photo-electrons from ionizations of neutral hydrogen (second term), neutral helium (third term), and singly ionized helium (fourth term).  Collisional ionizations due to thermal electrons in the gas are accounted for separately through the $\beta_i$ coefficients.

At each time-step, we also solve for the temperature evolution, $dT/dt$, which
is given by
\begin{equation}
    \frac{3}{2}\frac{d}{dt}\left(\frac{\kB T \nb}{\mu}\right) = \fheat \sum_i n_i \Heat_i - \sum_i \zeta_i n_e n_i - \sum_i \eta_i n_e n_i - \sum_i \psi_i n_e n_i \label{eq:TemperatureEvolution} 
\end{equation}
where $\Heat_i$ is the photo-electric heating rate, given by
\begin{equation}
    \Heat_i = \int_{E_i}^{\infty} \sigma_i(E) \spec(E, r)(E - E_i) \frac{dE}{E} \label{eq:HeatingRate} ,
\end{equation}    
and $\zeta_i$, $\eta_i$, and $\psi_i$ are the collisional ionization,
recombination, and collisional excitation cooling coefficients, respectively.
The constants in Equation \ref{eq:HeatingRate} are the total number density of
baryons, $\nb$, the mean molecular weight, $\mu$, Boltzmann's constant, $\kB$,
and the fraction of a photon's original energy deposited as heat, $\fheat$. In
the remaining sections we only include the effects of secondary electrons when
considering X-ray sources, which emit photons in the range $10^2\mathrm{eV} <
E < 10^4\mathrm{eV}$. In this regime, the values of $\fheat$ and $\fion$
computed via the formulae of \cite{Shull1985} are sufficiently accurate, but
for radiation at lower energies, the lookup tables of \cite{Furlanetto2010}
would be more appropriate.

An important simplification is that all radial and time dependencies can be
pulled outside of the integrals in Equations \ref{eq:PhotoionizationRate} and
\ref{eq:HeatingRate}, leaving (ignoring the secondary ionization term)
\begin{equation}
    \Gamma_i(n^{\mathrm{col}}) = A \int_{E_i}^{\infty} \sigma_i(E) I(E) e^{-\tau(E, n^{\mathrm{col}})} \frac{dE}{E} \label{eq:Gamma_final} ,
\end{equation}
and
\begin{equation}
    \Heat_i(n^{\mathrm{col}}) = A \int_{E_i}^{\infty} \sigma_i(E) I(E) (E - E_i) e^{-\tau(E, n^{\mathrm{col}})} \frac{dE}{E} \label{eq:Heat_final}
\end{equation}
where $A \equiv \Lbol / 4 \pi r^2$, and $n^{\mathrm{col}} =
(n_{\mathrm{HI}}^{\mathrm{col}}, n_{\mathrm{HeI}}^{\mathrm{col}},
n_{\mathrm{HeII}}^{\mathrm{col}})$.

From Equations \ref{eq:Gamma_final} and \ref{eq:Heat_final} it is clear that
the values of $\Gamma_i$ and $\Heat_i$ can be tabulated as a function of
column density for a given source. This greatly improves the speed of 1D
radiative transfer calculations, like those of \cite{Thomas2008}, since
integral values in Equations \ref{eq:HIIRateEquation},
\ref{eq:HeIIRateEquation}, \ref{eq:HeIIIRateEquation}, and
\ref{eq:HeatingRate} no longer need to be computed numerically on-the-fly. It
also provides the basis for our frequency resolution optimization strategy, as
presented in \S\ref{sec:Methods}. Note, however, that the dimensionality of
these lookup tables is equal to the number of absorbing species, so the tables
for simulations including only hydrogen are 1D, while those including helium
are 3D. If we choose to adopt the secondary electron treatment of
\cite{Furlanetto2010} our lookup tables inherit an additional dimension, as
the secondary ionization and heating factors $\fion$ and $\fheat$ depend on
both energy and ionized fraction, $x_i$. Since integration takes place over
energy, $x_i$ becomes the final dimension of the lookup tables.

For simplicity, our current treatment neglects cooling via free-free emission,
as well as cosmological effects, including Hubble cooling due to cosmological
expansion, Compton heating/cooling off cosmic microwave background (CMB)
photons, and photo-ionization by Wien-tail CMB photons.  


%%% 
%% Classes
%%%
\section{Classes}
In order of appearance in \texttt{RT1D.py}.

\subsection{\texttt{InitializeGrid}}
\subsection{\texttt{InitializeIntegralTables}}
\subsection{\texttt{RadiationSource}}
\subsection{\texttt{Radiate}}
\subsection{\texttt{SolveRateEquations}}
\subsection{\texttt{Interpolate}}
\subsection{\texttt{WriteData}}

%%% 
%% Parameters
%%%
\section{Parameters}

\subsection{Problem Types}
These were added for convenience in January, 2011.  If you want to make changes to these test problems you can, just make sure \texttt{ProblemType} is the first (uncommented) line in your parameter file.  Additional parameters afterwards will be interpreted as usual.

\begin{description}
    
\item [\texttt{ProblemType = 1}] This will set up a standard test with known analytic solution: the propagation of an ionization front in an isothermal, hydrogen-only, static universe.  The spectrum used is monochromatic at $E = 13.6 \ \text{eV}$, with photon luminosity $L_{\nu} = 5\times 10^{48} \ s^{-1}$.  The initial density, temperature, and box size are the same as the values used in Test 1 of Wise \& Abel 2010. 

\item [\texttt{ProblemType = 2}] This problem is almost exactly the same as \texttt{ProblemType = 1}.  However, the temperature is allowed to evolve, and instead of using a monochromatic spectrum, we use a $10^5$ K blackbody spectrum.  Again, the initial density, temperature, box size, and energy groups are the same as the values used in Test 1 of Wise \& Abel 2010.  

\item [\texttt{ProblemType = 2.1}] Same as \texttt{ProblemType = 2} for everything, except we sample the continuous blackbody spectrum with the four energy groups used by Wise \& Abel 2010.

\item [\texttt{ProblemType = 3.1}] Test problem \# 3 from \cite{Iliev2006} -- I-front trapping in a dense clump and the formation of a shadow.  The setup is the same as in \texttt{ProblemType = 2.1}, except a 1D clump is initialized at $x = 0.76 L_{\mathrm{box}}$, overdensity $\delta = 200$, radius $0.12$kpc, and temperature $T = 40$K, and the radiation field is plane-parallel.
\end{description}

\subsection{Initial Conditions}
\begin{description}
    
\item [\texttt{DensityProfile}]  

\item [\texttt{TemperatureProfile}]

\item [\texttt{InitialHIIFraction}]  

\end{description}


\subsection{Control Parameters}


\subsection{Source Parameters}


\subsection{Physics Parameters}


\section{Analysis}
Since our datasets are small, we can read in the entire time evolution all at once.  To do this, type the following in a Python terminal:
\begin{verbatim}
import rt1d.analysis as rta
import pylab as pl

ds = rta.Analyze('./pf.dat')                # load parameter file and data
pl.loglog(ds.data[10].r, ds.data[10].x_HI)  # radius vs. neutral fraction
\end{verbatim}

%%% 
%% Start to Finish
%%%
\section{Start to Finish}

%%% 
%% Convergence
%%%
\section{Convergence}
In \texttt{rt1d/doc/examples} you will find directories containing convergence test suites for RT06 test problems 1 \& 2 (\texttt{RT06\_1\_ResolutionTestSuite} and \texttt{RT06\_2\_ResolutionTestSuite}).  Each of these directories contains the parameter files necessary to run these problems on grids containing $100 \times 2^n$ cells, where $n = 0,1,...6$.  If you want, you can run them all at once, though keep in mind the simulations with 100 cells will finish much much faster than those containing 6400 cells, so it might not be good to check out a node for this as most of the computation will be spent on 1 or 2 simulations.  For instance, you could do:
\begin{verbatim}
    cd rt1d/doc/examples/RT06_1_ResolutionTestSuite/c_infinite
    mpirun -np 4 python RT1D.py -b batch_list.dat
\end{verbatim}
I've included the same setup for $c \neq \infty$ in \texttt{rt1d/doc/examples/RT06\_1\_ResolutionTestSuite/c\_finite}, so you can see the differences (if any) caused by making the $c \rightarrow \infty$ approximation.


\newpage
\bibliography{references}
\bibliographystyle{plain}





\end{document}



