%%% rt1d

\documentclass[letterpaper,titlepage,12pt]{article}

%%% Preamble 

\setlength{\topmargin}{0in}
\setlength{\oddsidemargin}{0in}
\setlength{\evensidemargin}{0in}
\setlength{\textwidth}{6.5in}
\setlength{\textheight}{9in}
\setlength{\headheight}{0in}
\setlength{\headsep}{0in}
\setlength{\marginparsep}{0in}
\setlength{\marginparwidth}{0in}

\usepackage{amsmath, amsthm, amssymb}
\numberwithin{equation}{section}
\usepackage{graphicx}
\usepackage{pslatex}

%%% Beginning of Document

\begin{document}
	
\author{Jordan Mirocha}	
	
\title{\Large {\bf rt1d: User's Manual}}
\date{Last Updated: \today}
\maketitle

\setcounter{tocdepth}{2}
\tableofcontents
\newpage

%%% 
%% Methods
%%%
\section{Methods}

%%% 
%% Classes
%%%
\section{Classes}
In order of appearance in \texttt{rt1d.py}.

\subsection{\texttt{InitializeGrid}}
\subsection{\texttt{InitializeIntegralTables}}
\subsection{\texttt{RadiationSource}}
\subsection{\texttt{Radiate}}

%%% 
%% Start to Finish
%%%
\section{Start to Finish}

%%% 
%% Parameters
%%%
\section{Parameters}

\subsection{Problem Types}
These were added for convenience in January, 2011.  If you want to make changes to these test problems you can, just make sure \texttt{ProblemType} is the first (uncommented) line in your parameter file.  Additional parameters afterwards will be interpreted as usual.

\begin{description}
    
\item [\texttt{ProblemType = 1}] This will set up a standard test with known analytic solution: the propagation of an ionization front in an isothermal, hydrogen-only, static universe.  The spectrum used is monochromatic at $E = 13.6 \ \text{eV}$, with photon luminosity $L_{\nu} = 5\times 10^{48} \ s^{-1}$.  The initial density, temperature, and box size are the same as the values used in Test 1 of Wise \& Abel 2010. 

\item [\texttt{ProblemType = 2}] This problem is almost exactly the same as \texttt{ProblemType = 1}.  However, the temperature is allowed to evolve, and instead of using a monochromatic spectrum, we sample a $10^5$ K blackbody with four energy groups.  Again, the initial density, temperature, box size, and energy groups are the same as the values used in Test 1 of Wise \& Abel 2010.  

\item [\texttt{ProblemType = 2.1}] Same as \texttt{ProblemType = 2} for everything, except we use a continuous blackbody spectrum rather than a discrete one.

\end{description}

\subsection{Initial Conditions}
\begin{description}
    
\item [\texttt{DensityProfile}]  

\item [TemperatureProfile]  

\end{description}


\subsection{Control Parameters}


\subsection{Source Parameters}












\end{document}



